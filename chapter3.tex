% chap3.tex (Definitions and Theorem)

\chapter{Model Performance with Regularization and Data Augmentation}

Regularization is any approach used to reduce test error but not training error. One way
to regularize a model is to introduce bias and reduce variance by imposing any constraint on
its weights. By adding a penalty term to the cost function, we can shrink weights to avoid
them from blowing up in magnitude and overfitting to the training set. L2 regularization
adds the L2 norm of the weights to the cost function:

\[J(\bm{\theta}) = -\mathbb{E}_{\bm{x},y \sim \hat p_{data}} \log \textit{p}(y|\bm{x};\bm{\theta}) + \lambda \lVert \bm{\theta} \rVert_{2}\]

where $\lambda$ is the regularization control parameter. Higher values of $\lambda$ penalize more, where
a value of 0 degenerates to no regularization.
A more recent and highly effective way to regularize a neural network is via \textbf{dropout}.
Dropout "deactivates" a neuron or hidden unit with probability $p_{drop}$. In practice, a unit’s
output is set to 0 with probability $p_{drop}$.

Another approach that has received much recent attention is to augment the data set.
\textbf{Data augmentation} refers to applying transformations to the input data in a way that
the label value does not change.


\section{Dropout}
Dropout is one of the most popular regularization mechanism used in deep networks today.
Simply put, dropout \textit{deactivates} a unit with probability $p_{drop}$. This forces the network to not rely on certain
activations, since any unit will only be \textit{active} with probability $1-p_{drop}$[CITE].

[IMAGE]

\section{Dataset Augmentation}
Data augmentation is ubiquitous in computer vision tasks. Example transformations include
small translations, rotations, and even color intensity jitters via Pricipal Component
Analysis[CITE]. In our work, we use sequence data i.e. scientific publication abstract texts.

\subsection{Zero-Padding: Uniform Input Length}
Although more complex neural models are designed to cope with variable length input,
in practice a more common and simple approach is to pad data to be of some specified
length as described in chapter [CITE CHAPTER]. The input length is a hyperparameter
that should be fine-tuned, but a reasonable approach is to pad enough to fully accomodate
the length of most input sequences i.e. it’s preferrable to pad than truncate and lose
information.

\subsection{Padding with Informative Values}
When we pad our input sequence with 0’s, we don’t add any additional information; we
simply create a constant input length. If instead we add values that characterize or help
describe the input sequence more thoroughly, we increase the amount of useful information
available during neural network training.

Consider a very simple input sequence, and assume its true label can be determined
from a single word:

\[\text{"This paper is about computer graphics"}\]

where the label is "Graphics" i.e. a publication about computer graphics.
In this simple case, the label is determined by the word "graphics". A single, \textbf{very
informative} word is enough for the classifier to predict the label correctly. This very
informative word however is only 1/6 of the entire text.

Now consider a the padded version, where we enfore an input length of 10:

\[\text{"This paper is about computer graphics PAD PAD PAD PAD"}\]

In the padded version, the word graphics is now only 1/10 of the entire text. If we
could find this very informative word from within the text and pad using it instead of some
meaningless token e.g. 0's, we could increase the information of this word over the entire
input sequence by making it comprise a larger amount of the input sequence. In other
words, make important words be present more frequently.

We propose to pad input sequences with \textbf{meaningful} values found already within the
input text sequence instead of 0's only. In the following section, we will develop on how to
extract these meaningful words via a cosine similarity-based metric and apply them to pad
our data.


\section{Keyword Extraction: Cosine Similarity Optimization (CSO)}
As described in [CITE SECTION], we use a token-to-embedding dictionary to map an
input sequence of word index i.e. integers into an embedding matrix i.e. a sequence of
embedding vectors.
This vectorial represention of words can be obtained after learning a language model
using a neural network[CITE]. One property of this word embedding space is that words
can now be represented as dense, continuous vectors of dimensionality much smaller than
vocabulary size. Another important property is that semantic relationships are captured
in this embedding space and can be measured between any two words using the cosine
similarity metric[CITE]. This is an attractive property that allows for term comparison
by semantic relationships. These relationships can range from complete similarity i.e. the
same term, to term orthogonality i.e. unrelated words, and even capture the direction of
the relation e.g. negative values of cosine similarity.

The cosine similarity between two vectors can be computed as follows:

\[
  \frac{\mathbf{x} \cdot \mathbf{y}}{\lVert \mathbf{x} \rVert \lVert \mathbf{y} \rVert}
\]

We can therefore quantify similarity between words in a sentece as a function of their
cosine similarity with respect to each other. We refer to this measure as the \textit{within-sentence}
cosine similarity. Similarly, we can measure this similarity for words within a noun phrase
in a sentence. We call this the \textit{within-noun-phrase} cosine similarity.

\subsection{Extracting Keywords: Pipeline}
Given input text sequence: Given a document, our keyword extraction algorithm works as
follows:
\begin{itemize}
  \item{Remove non-alphanumeric characters and convert to lowercase}
  \item{Tokenize each sentence into a list of words}
  \item{Perform Part of Speech tagging on each list of words i.e. sentence}
  \item{Chunk each sentence into a set of noun phrases}
  \item{Map each word in each noun phrase to its corresponding word embedding vector}
  \item{Perform Cosine Similarity Optimization (CSO) to extract a keyword noun phrase for each sentence}
\end{itemize}

\subsection{Extracting Keywords: Algorithm}

The procedure CSO receives as input a set of noun phrases $np^i = \{np^{i}_{1}, ..., np^i_{|\bm{np}^i|}\}$ corresponding
to a sentence, where $np^{i}_{k}$ is the $k$th word embedding in the sentence’s $i$th noun phrase.

Given an input chunked sentence, CSO computes the average cosine similarity between
all words in a noun phrase, for all noun phrases in the sentence. The algorithm returns the
noun phrase which \textbf{optimizes} its average cosine similarity metric.

The reasoning behind the cosine similarity minimization is as follows: An informative
noun phrase within a sentence will contain a word which will stand out from the rest of the
words in the noun phrase. We quantify this notion of \textit{standing out from the rest} using the
mean within-noun-phrase cosine similarity. We therefore compute, for all noun phrases in a
sentence, the mean \textit{within-noun-phrase} average cosine similarity, as described in [REFER TO ALGORITHM].
We select the noun phrase that minimizes this measure.

In practice, we find both maximization and minimization of the mean \textit{within-nounphrase}
average cosine similarity leads to interesting results i.e. reasonable words to be
considered keywords, so we include them both for padding.
[INCLUDE SNAPSHOT OF CSO OUTPUT]

\subsection{Proposed Padding Scheme}
As described in [CSO SECTION], can use a cosine similarity-based metric to extract
keywords from text. Consider an input text sequence $S$ that yields keyword indexes
$BoW(CSO(S)) = x_{key^1}, ..., x_{key^j}$ and input $\bm{x}_{pad} = x_1, ..., x_k, 0_{k+1}, ..., 0_n$, we concatenate
the non-zero entries $x_1, ..., x_k$ with the extracted keyword indexes:

\[\bm{x}_{cso} = x_1, ..., x_k, x_{key^1}, ..., x_{key^j}, 0_{k+j+1}, ..., 0_n\]


\section{Dataset Augmentation: Padding with Similar Keywords}
Our CSO padding scheme can only pad with words already present in the input
text sequence. It could prove beneficial to further pad the input with words that are \textit{similar}
to its keywords i.e. words extracted via CSO. We therefore further pad the input sequence
with the nearest neighbor of each keyword. In this context, we consider the nearest neighbor
of a word embeddings $\bm{w}$ as:

\[\argmin_{\bm{w}_i} 1 - \frac{\bm{w} \cdot \bm{w_i}}{\lVert \bm{w} \rVert \lVert \bm{w_i} \rVert}\]

We use a Locality Sensitive Hashing forest for our approximate nearest neighbor search[CITE].
