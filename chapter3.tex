% chap3.tex (Definitions and Theorem)

\chapter{Towards Improving Model Performance: Regularization and Data Augmentation}

\section{Model Regularization}
Regularization is any approach used to reduce test error but not training error.
[ELABORATE]

\subsection{Weight Regularization}
One way to regularize a model to impose a constraint on its weights. By adding a penalty term to the cost function, we can shrink weights to avoid
them from blowing up in magnitude and overfitting to the training set. L2 regularization
adds the L2 norm of the weights to the cost function:

\[J(\bm{\theta}) = -\mathbb{E}_{\bm{x},y \sim \hat p_{data}} \log \textit{p}(y|\bm{x};\bm{\theta}) + \lambda \lVert \bm{\theta} \rVert_{2}\]

where $\lambda$ is the regularization control parameter. Higher values of $\lambda$ penalize more and a value of 0 leads to no regularization.

If the embedding layer's weights are allowed to be fine-tuned during training, it is
important to regularize them. Because of the large amount of parameters in our model's embedding layer (number of words times embedding size),
overfitting occurs quite rapidly if the weights are allowed to grow without constraint.

\subsection{Dropout}
A more recent and highly effective way to regularize a neural network is via \textbf{dropout}.
Dropout "deactivates" a neuron or unit with probability $p_{drop}$. We deactivate a unit
by setting its output to 0. This forces the network to not rely on certain
activations, since any unit will only be \textit{active} with probability $1-p_{drop}$[CITE].

[IMAGE]
\section{Dataset Augmentation: Shuffling and Noise Injection}
Another way to improve model performance is to \textbf{augment} the dataset.
Data augmentation refers to any transformation of the input data in a way that
the label value does not change. Data augmentation is ubiquitous in computer vision tasks. Example transformations include translations, rotations, and even color intensity jitters via Pricipal Component
Analysis[CITE]. All these tranformations should be are subtle enough that the overall structure is preserved, but
allow for the model to process more distinct training data points.

In order to introduce more variance into our dataset, we propose to \textbf{shuffle} and \textbf{add noise} to the input sequences.
We shuffle by randomly permuting non-overlapping neighborhoods in the input text sequence i.e.\textbf{context windows}.
We further propose to add small amounts of
noise to the input sequence by randomly replacing words with words taken from the training vocabulary.

\subsection{Shuffling}
We propose to augment our dataset by making small \textit{context} changes that don't change the global structure
of the input sequences. Concretely, we move a non-overlapping \textit{context window} along the input sequence, randomly
shuffling the words indexes inside it. For example a context window of size 2 i.e. \textit{bi-gram}, an input sequence
\[\bm{x} = x_1, x_2, x_3, x_4, x_5, x_6\]

 could be shuffled to:
\[\bm{x}\prime = x_2, x_1, x_3, x_4, \underbrace{x_6, x_5}_\text{bi-gram}\]
In the first pass, the first two indexes get shuffled. The second pass led to no changes in ordering. The third and
final pass shuffles the last two indexes.

The motivation for this dataset augmentation technique is that small changes in the ordering of the words will
simulate a larger training sample; a training input sequence has a low chance of being repeated as its
length increases. Smaller context windows preserve the most structure. This method will preserve
most of the original context structure in the sequence, as long as the shuffling is not to harsh e.g.
a complete random permutation of the sequence.

\subsection{Noise Injection}
We propose to augment our dataset by injecting small amounts of noise to each training sample. Again, we aim to simulate
a larger training sample while avoiding harsh changes to the original inputs.
\begin{algorithm}[H]
\caption{Add noise to input sequence}
\begin{algorithmic}[1]
\Procedure{NoiseInjection}{$\bm{x}=[x_1,...,x_n], \mathbb{V}, p_{noise}$}
\For {$k$ = 1 to $n$}
\If{$p_k \sim \textit{U}(0,1) \leq p_{noise}$}
\State$x_k \gets x^{\prime}_k \in \mathbb{V}$
\EndIf
\EndFor

\Return $\bm{x}$
\EndProcedure
\end{algorithmic}
\end{algorithm}


\section{Dataset Augmentation: Padding}

Although more complex neural models are designed to cope with variable length input,
in practice a more common and simple approach is to pad data to be of some specified
length as described in chapter [CITE CHAPTER].
In order to enforce uniform input size for our neural networks,
we apply \textbf{zero-padding}.For any arbitrary training instance $BoW(\bm{s})=$ $\bm{x} = x_1,...,x_k$, we enforce that $k = n$, for the specified input size
$n$. Thus, if $k \textless n$, we transform it into $\bm{x}_{pad} = x_1,...,x_k, 0_{k+1}, ..., 0_{n}$. Conversely, if
$k \textgreater n$, we simply truncate $\bm{x}$ to be of size $n$. The input length introduces another model hyper-parameter
that should be fine-tuned, but a reasonable approach is to pad enough to fully accomodate
the length of most input sequences i.e. it’s preferrable to pad than truncate and lose
information.

Our network's input is a sequence of integers, each integer being a word index: a number representing a word in our vocabulary $\mathbb{V}$.
Word indexes range from 1 to $|\mathbb{V}|$, and the index 0 being left for out-of-vocabulary words.
When we pad our input sequence with 0’s, we don’t add any additional information; we
simply create a constant input length. We propose that if instead we add values that characterize or help
describe the input sequence more thoroughly, we may increase the amount of useful information
available during neural network training.

Consider a very simple input sequence, and assume its true label can be determined
from a single word:

\[\text{"This paper is about computer graphics"}\]

where the label is "Graphics" i.e. a publication about computer graphics.
In this simple case, the label is determined by the word "graphics". A single, \textbf{very
informative} word is enough for the classifier to predict the label correctly. This very
informative word however is only 1/6 of the entire text.

Now consider a the padded version, where we enfore an input length of 10:

\[\text{"This paper is about computer graphics PAD PAD PAD PAD"}\]

In the padded version, the word graphics is now only 1/10 of the entire text. If we
could find this very informative word from within the text and pad using it instead of some
meaningless token e.g. 0's, we could increase the information of this word over the entire
input sequence by making it comprise a larger amount of the input sequence. In other
words, make important words be present more frequently.

We propose to pad input sequences with \textbf{meaningful} values found already within the
input text sequence instead of 0's only. In the following section, we will develop on how to
extract these meaningful words via a cosine similarity-based metric and apply them to pad
our data. In the following subsections, we propose several padding schemes in order reduce the amount of non-informative
when training.

\subsection{Wrap Padding}
Our first padding scheme is to \textit{wrap around} the text, repeating words once we reach the padding portion.
Refering back to the first example, the input sequence:
\[\text{"This paper is about computer graphics PAD PAD PAD PAD"}\]

would then be
\[\text{"This paper is about computer graphics This paper is about"}\]

One thing to observe is that using this simple scheme, we remove all 0's i.e. non-informative padding indexes
from the text sequence, but we also don't have selection mechanism and can miss useful words such as missing the
word "\textit{graphics}" in this example. Nonetheless, it is a simple first approach to pad our data.

\subsection{Keyword Padding}
One characteristic of the wrap padding scheme described above is that there is no selection mechanism. Words are not selected based on some metric,
but rather just \textit{wrapped}. We now propose a simple algorithm to select \textit{keywords} from within the input text sequence
based on a cosine similarity based metric, as well as a padding scheme using those keywords.

Using word embeddings, word tokens i.e. strings can now be represented as dense, continuous vectors of dimensionality much smaller than
vocabulary size. Another useful property of embeddings is that semantic relationships are captured
in this embedding space and can be measured between any two words using the cosine
similarity metric[CITE]. This is an attractive property that allows for term comparison
by semantic relationships. These relationships can range from complete similarity i.e. the
same term, to term orthogonality i.e. unrelated words, and even capture the direction of
the relation e.g. negative values of cosine similarity.

The cosine similarity between two vectors can be computed as follows:
\[
  \frac{\mathbf{x} \cdot \mathbf{y}}{\lVert \mathbf{x} \rVert \lVert \mathbf{y} \rVert}
\]
We can therefore quantify similarity between words in a sentece as a function of their
cosine similarity with respect to each other. We refer to this measure as the \textit{within-sentence}
cosine similarity. Similarly, we can measure this similarity for words within a noun phrase
in a sentence. We call this the \textit{within-noun-phrase} cosine similarity.
We represent a noun phrase $\mathbf{np}$ as a matrix composed of horizontally stacked vectors of size $d$, each vector a word in the noun phrase:

\[\mathbf{np} = \begin{bmatrix}
    \bm{np}_{1}   \\
    \vdots \\
    \bm{np}_{n}
  \end{bmatrix},
\bm{np}_{i} \in \mathbb{R}^{1 \times d}\]

Given a noun phrase $\mathbf{np}$ comprised of $n$ unit norm embeddings $\bm{np}_i$, we can compute its \textit{within noun phrase} cosine similarity by:

\[c = \frac{1}{{n^2}} \sum_{i=1}^{n} \sum_{j=1}^{n} \begin{bmatrix}
    \bm{np}_{1}   \\
    \vdots \\
    \bm{np}_{n}
  \end{bmatrix} \times \begin{bmatrix}
      \bm{np}_{1}   \\
      \vdots \\
      \bm{np}_{n}
    \end{bmatrix}^\mathsf{T}\]

Thus, given a set of noun phrases $\{\mathbf{np}^{(1)}, \dots \mathbf{np}^{(m)}\}$ from a sentence, we compute the corresponding set of
\textit{within noun phrase} cosine similarities $\bm{c} = \{c_1 \dots c_m\}$, and we select the noun phrase which corresponds to minimum

\[\mathbf{np}^{(j)}, j = \argmin{\bm{c}}\].

The reasoning behind the cosine similarity minimization is as follows: An informative
noun phrase within a sentence will contain a word which will stand out from the rest of the
words in the noun phrase. We quantify this notion of \textit{standing out from the rest} using the
mean within-noun-phrase cosine similarity. We therefore compute, for all noun phrases in a
sentence, the mean \textit{within-noun-phrase} average cosine similarity, as described in [REFER TO ALGORITHM].
We select the noun phrase that minimizes this measure.

In practice, we find both maximization and minimization of the mean \textit{within-nounphrase}
average cosine similarity leads to interesting results i.e. reasonable words to be
considered keywords, so we include them both for padding.
[INCLUDE SNAPSHOT OF CSO OUTPUT]

\subsection{Proposed Padding Scheme}
We propose to pad an input sequence $\bm{x}_{pad} = x_1, ..., x_k, 0_{k+1}, ..., 0_n$ using
the indexes of extracted keywords $x_{key^1}, ..., x_{key^j}$ by concatenating
the non-zero entries $x_1, ..., x_k$ with the extracted keyword indexes:

\[\bm{x}_{cso} = x_1, ..., x_k, x_{key^1}, ..., x_{key^j}, 0_{k+j+1}, ..., 0_n\]


\section{Dataset Augmentation: Padding with Similar Keywords}
Our CSO padding scheme can only pad with words already present in the input
text sequence. It could prove beneficial to further pad the input with words that are \textit{similar}
to its keywords i.e. words extracted via CSO. We therefore further pad the input sequence
with the nearest neighbor of each keyword. In this context, we consider the nearest neighbor
of a word embeddings $\bm{w}$ as:

\[\min_{\bm{w}_i} 1 - \frac{\bm{w} \cdot \bm{w}_i}{\lVert \bm{w} \rVert \lVert \bm{w}_i \rVert}, \bm{w} \neq \bm{w}_i\]

We use a Locality Sensitive Hashing forest for our approximate nearest neighbor search[CITE].
After having computed the nearest neighbor of each keyword, we pad as follows:

\[\bm{x}_{lsh} = x_1, ..., x_k, x_{key^1}, ..., x_{key^j},x_{lsh^1}, ..., x_{lsh^j}, 0_{k+j+1}, ..., 0_n\]

\section{Reducing Data Granularity: From Abstract to Sentences}
Our last proposed dataset augmentation scheme is to consider the data as a set of sentences rather than abstracts.
\subsection{Voting Schemes}
Here we introduce our proposed voting schemes
