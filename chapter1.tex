% chap1.tex {Introductory Chapter}

\chapter{Introduction}

\section{Brief Overview of Deep Neural Networks}
Deep convolutional neural networks have seen an enormous amount of success on a wide
array of application, from scene interpretation to self-driving vehicles and art generation[CITE].
Natural language processing tasks are no exception to the range of problems
deep learning can solve, from sentiment analysis to language modeling. In this work, we
focus our efforts to studying convolutional neural networks for text classification. Especifically,
we analyse how well modern neural networks models performed using scientific
abstract text data from multiple disciplines such as astro-physics and computer science.
The term modern in this context refers to neural models that use popular and recently
(re)discovered techniques to achieve state-of-the-art performance on most machine learning
benchmark datasets[CITE].

\section{Training a Neural Network}

A neural network is a function $f(\bm{x};\bm{\theta})$ that maps its input $\bm{x}$ to some response variable $y$. When we \textit{train} a
neural network, we \textit{learn} the model parameters, or weights, $\bm{\theta}$ that minimize some cost function $J(\bm{\theta})$.
For a regression task, where the model's output is a continuous variable, a common cost function is the \textbf{Mean Square Error}:
\[J(\theta) = \frac{1}{m}\sum_{i=1}^{m}(y_{i} - f(\bm{x}_{i};\bm{\theta}))^{2}\]

For categorical or discrete output variables found in e.g. classification tasks, we use the \textbf{Categorical Cross-Entropy}:

\[J(\bm{\theta}) = -\mathbb{E}_{\bm{x},y \sim \hat p_{data}} \log \textit{p}(y|\bm{x};\bm{\theta})\]

Given a \textit{training} set of observations $\bm{x}_i$ and their true labels $y_i$, we compute weights that minimize the cost, or error, via
maximum likelihood (ML) estimation:

\[\bm{\theta}_{ML} = \argmax_{\bm{\theta}} \sum_{i}^{m} \log P(y_{i}|\bm{x}_{i};\bm{\theta})\],

which one can see is the equivalent of computing the weights that \textbf{\textit{minimize}} the cross-entropy cost function.


%\section{Minimizing Non-Convex Functions: Gradiant-Based Learning}
%Convex functions are easy to optimize. Their global minimum is guaranteed and any local minimum is a global minimum. This means that when a
%procedure optimizes a convex function, its solution is truly optimal[CITE]. Said procedure is also guaranteed to converge from any initial point in parameter space[CITE].
%Due to the non-linearities associated with a neural network, the loss function to be minimized becomes non-convex.
%[Elaborate (Weight symmetry, model identifiability, etc)]

%Because of the non-convexitivity of neural network optimization, we usually rely on gradient-based methods. This involves iteratively \textit{moving},
%or updating, the values of our weights in the direction opposite of the cost function's gradient. These updates should be small, scaled by a
%\textit{learning rate}, and can make learning a lengthy process. Nonetheless, this method is effective and allows for learning of a good
%set of weights, although in practive it is rarely the optimal[CITE].

\section{Bias-Variance Tradeoff}
When learning a neural network's weights, we use a \textit{training set} so that we can later generalize
previously unseen data with high accuracy (or some other determined metric). That means that during training, we
obtain $\bm{\theta}_{ML}$ by minimizing $J_{train}(\bm{\theta})$, but we care about having low $J_{test}(\bm{\theta})$ i.e. low cost
on test data points.

\textbf{Overfitting} occurs when a network is able to predict its training set extremely well i.e. very close to zero error, but fails to predict
unseen data points. This is because the network's weights have been extremely fine-tuned to \textit{fit} its training data, but do not fit or represent
data points outside of its training sample.
An overfitted model is said to have large \textbf{variance} and small \textbf{bias}. Conversely, underfitting occurs when the model fails to predict
the training set because it generalizes too harshly. This model is said to have large bias and small variance.

Because of the commonly large amount of weights in deep convolutional networks, it is easy to overfit even a moderate size training set[CITE].
In this work, we study the effects of multiple regularization techniques used to avoid overfitting in a neural network[REFER TO CHAPTER].
