% abstract.tex (Abstract)

\addcontentsline{toc}{chapter}{Abstract}

\chapter*{Abstract}
Convolutional neural networks have seen much success in computer vision and natural language processing tasks.
Recent convolutional network architectures employ word embeddings, or words represented as dense vectors, to
transform a given input sequence of words into a dense matrix of continuous values akin to image data.
This allows for the well known convolution/pooling operations to be applied to text data in a manner similar to
convolution and pooling operations on image data. These embeddings may incorporated into the neural
network itself as a trainable layer to allow fine-tuning their values to better fit the data.
Because of the large number of parameters in these models', they are prone to overfitting
if not regularized appropriately or trained on a sufficiently large dataset.
In this work, we study the performance of convolutional neural networks when the available training datasets
are of moderate size and propose several data augmentation techniques designed for text datasets in an attempt
to mitigate overfitting and improve model generalization.
Finally, we provide discussion on our empirical results as well as future work.
