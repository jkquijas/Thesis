% chap2.tex (Definitions)

\chapter{Text Classification with Deep Neural Networks}\label{TXT-CLASS}

\section{Brief Overview of Deep Learning}
Deep convolutional neural networks have seen much success on a wide
array of application, from scene interpretation to self-driving vehicles and art generation.
Natural language processing tasks are no exception to the range of problems
deep learning can solve, from sentiment analysis to language modeling. This success of deep neural networks
is accredited to their ability to learn high-level features from raw input data\cite{le2013building}.

One type of neural networks that is particularly good at learning features from the data is the convolutional neural network (CNN)\cite{dosovitskiy2014discriminative}\cite{lecun1989backpropagation}\cite{oquab2014learning}.
A CNN learns a set of weights, or convolution kernels, that are normally much smaller than the input size.
The difference in size forces the weights to have \textbf{sparse interaction}, or interaction on a subregion of the input, versus the dense interaction
between every input with every output in traditional neural networks (vector-matrix multiplication). This sparse interaction
allows for features to be detected locally within the input (for example, an edge detector).
\begin{figure}[H]
\centering
\includegraphics[width=1.0\textwidth]{SparseInteraction.png}
\caption{Visualization of sparse interaction (left) and dense interaction (right) between inputs and weights. With sparse interaction, the weights only
interact on a subregion of the entire input. This is in contrast to the dense interaction of classic neural networks, where there's an interaction between every
input and every output.}
\end{figure}
\textbf{Parameter sharing} allows for distinct outputs to be computed using the same set of weights. This drastically reduces
the number of unique weights and allows for significant increases in network depth (i.e. deep networks) without the need
to increase the amount of training data.
These two design principles make CNN's poweful and efficient feature extractors.

Another type of deep neural network that has seen much success is the recurrent neural network (RNN). RNN's are designed to model data that display temporal structure, such as text and speech. In contrast to
traditional neural networks, by \textit{unfolding}
time-steps as a computational graph, the network can learn to model data along the time dimension, using at each time step the
raw input plus the output of the last time step's layer \cite{siegelmann1995computational}. Because of the usually very deep structure of
the recurrent network, computation of the gradient via back propagation can lead to the \textit{vanishing gradient} problem, where the gradient
starts to shrink and become very close to zero. The \textbf{Long Short-Term Memory (LSTM)} recurrent network proposed by \cite{hochreiter1997long} essentially solves this problem.
A recent and simpler variation of the LSTM network is the Gated Recurrent Unit (GRU), which performs comparably well \cite{chung2014empirical}.


\section{Word Embeddings}
Another significant achievement of deep learning for natural language processing is the neural probabilistic language model. In their work, \cite{bengio2003neural}
proposed a neural network to model the probability of a word given its context
$P(w_t|w_{t-k},\dots,w_{t-1})$.

The curse of dimensionality, manifested by the fact that a test word sequence $(w_{t-k},\dots,w_{t})$
is likely to be different from all training sequences, was addressed by learning distributed representations of words. Words are represented as
dense continuous vectors called \textbf{word embeddings}.
Using word embeddings to model the likelihood of word sequences allowed generalization because the language model
would give high probability to unseen word sequences if they were made up of words semantically similar to already seen words.
Because of the underlying algorithm used to learn these word embeddings, similar words tend to lie closer to each other in
embedding space. Thus, word embeddings are said to capture semantic relations and encode more information than just a word identifier (e.g.
Bag of Words input representation).

\begin{figure}[h]
\centering
\includegraphics[width=1.0\textwidth]{EmbeddingViz.png}
\caption{Visualization of embeddings using the t-distributed stochastic neighbor embedding (t-SNE) dimensionality reduction algorithm.
Words with similar or related meanings tend to lie close to each other in embedding space.}
\end{figure}

\section{Convolutional Neural Networks}
As mentioned previously, convolutional neural networks are known for their abilities to learn high-level features from raw data. As input signals advance
forward through the network, they produce latent signals as linear combinations with learned parameters, have non-linearities applied
to them, and have a pooling or selection mechanism based on simple functions such as the average or maximum operations \cite{zhou1988image}.

When dealing with image data, images are convolved with multiple filters, each convolution applied to overlapping subimages called receptive fields.
This localized convolution process leads to discovery of low level features of images in the training set such as edges. As data
flows forward through the model, higher level features are discovered (e.g. wheels or headlights in a vehicle image dataset).

These convolutional neural networks are comprised of \textit{feature maps}. A feature map is a \textbf{convolution} layer paired with a
\textbf{pooling} layer afterwards. The convolution stage creates \textit{activations} by convolution of kernels with the inputs followed by a non-linear.
The convolution operator of two-dimensional data such as images can be defined as:

\[\begin{aligned}
S(i,j) = (I \ast K)(i,j)
= \sum_{m} \sum_{n} I(i-m,j-n)K(m,n)
\end{aligned}\]
where $m$ and $n$ are the kernel's width and height respectively.




\subsection{Input Representation: Word Embeddings for Convolutional Networks}
We can take advantage of word embeddings to apply convolutions to text in a fashion similar to convolutions with
image data and exploit the semantic information in the embeddings \cite{kim2014convolutional}. We apply the convolutions to overlapping sub-regions of the input text i.e. bi-grams, tri-grams, etc.
After convolution, we apply a non-linear function such as $max(0,x)$ and reduce data dimensionality
by pooling such as $x_{pool} = max(x_1,\dots,x_n)$.
Given a set of documents $\bm{s_1},...,\bm{s_m}$, we build a vocabulary $\mathbb{V}$ and a bag-of-words model from $\mathbb{V}$ to create a mapping $BoW:\mathbb{V} \mapsto \{1,...,|\mathbb{V}|\}$.
We represent a training document $\bm{s}$ as a sequence of integers $BoW(\bm{s})=$ $\bm{x} = x_1,...,x_k$, each integer being simply a word index
in $\mathbb{V}$. When the convolutional network receives this input sequence, each word index will be mapped to a corresponding embedding vector.

\begin{figure}[H]
\centering
\includegraphics[width=0.7\textwidth]{FeatureMap.png}
\caption{Visualization of a feature map with a single kernel. In this example, the kernel convolves tri-grams, or windows of three consecutive words.
After convolution with all possible trigram context windows, max pooling is applied to reduce dimensionality.
Here, the pool size is 3. This process is repeated as many times as there are kernels in the layer and their outputs are
concatenated horizantally to yield a matrix of size \textit{num filters} $\times$ (\textit{input length - pool size + 1}).}
\label{fig:convolution}
\end{figure}
%%% Local Variables:
%%% mode: latex
%%% TeX-master: "thesis"
%%% End:
