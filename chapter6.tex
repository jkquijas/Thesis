% chap6.tex (Significance and Future Work)

\chapter{Concluding Remarks}
We presented an evaluation of recurrent convolutional neural networks for classification of publication abstracts from multiple science and
engineering disciplines. Using our proposed techniques we obtained a small increase in test accuracy compared to our baseline model.

One conclusion we can make from this study is that recurrent convolutional networks are efficient and effective feature extractors even with
moderate amounts of training data. The combination of convolution and global max pooling provided efficient feature extraction, while the recurrent layer
learned to model the temporal structure of our data. The model with this combination of layers reached the highest test accuracy in our experiments.

We proposed several embeding \textit{trainability} schemes to manipulate the word embeddings during training.
The results we obtained were different than what we had anticipated. Since the embeddings were already learned (pre-trained) from a previous task to represent words in a distributed way and to capture semantic relations,
we originally believed that changing their values would only overfit the training set and thus significantly reduce testing set accuracy.
We expected that the \textit{freeze-unfreeze} method would achieve faster convergeance and higher accuracy compared to all other proposed embedding methods.
With this scheme, we hoped the model would converge then increase accuracy by allowing the embeddings to be trainable until convergeance determined by
a stopping criterion. Although the model test accuracy did increase with this scheme, the \textit{flash-freeze} method outperformed all others.
By allowing the embeddings to be fine-tuned for a small number of iterations (three in our tests), then allowing the model to rely
only on its weights (frozen embeddings), we observed we could finish training after a constant number of epochs and remove the need for a validation set.
For our task, three epochs were enough to finish termination. This increased, albeit by not much, the amount of data available for training purposes.
A validation set for early stopping was crucial before we incorporated the \textit{freeze-flash} stopping mechanism; the model would quickly overfit and test accuracy
would decrease.

One interesting observation from our experiments is that the use of PCA on the embeddings yielded favorable results. The dimensionality reduction and linear decorrelation of the embedding
variables did not affect the model's generalization when a large enough variance percentage was retained. This allowed us to add a smaller number of free parameters
while retaining competent performance compared to using the entire embedding set.

Our tests were inconclusive in determining whether the proposed noise injection and word shuffling augmentations yielded improved results. Overall, we obtained slightly
lower test accuracy with models trained using these augmentation schemes. We believe that the
model learns the structure of the data and therefore develops a sort of translation invariance. This could explain why shuffling the order of words as a data augmentation
technique yielded results similar than without the shuffling.
A larger dataset and more tests are necessary to decide if this technique is indeed beneficial or not.

We proposed to split the abstracts into sentences and train the model using this finer data granularity. This would increase the amount of training examples
the model could perform weight updates with. By increaseing the number of examples available for training, we hoped to obtain better test accuracy.
Our results were contraty to this, as we obtained higher accuracy when we treated an entire abstract as a single training example.
We believe this is due to the noise present in our data, as well as the increased likelihood of class ambiguity when processing a single sentence versus an entire
paragraph.

\section{Future Work}
We would like to extend our research to incorporate datasets of various sizes on our same task. This will give us further insight on how
well recurrent convolutional neural networks learn with a range of different training set sizes. Another further augmentation technique we would like to further
study is padding.

The application of PCA to the embedding matrix yielded unexpected and favorable results. Because of the projection of the embeddings into a different set of axes,
we believed we would lose the performance gain of using pre-trained embeddings. The results were contrary to this, as we obtained similar performance
withe the PCA embeddings with a smaller number of free parameters. It would be interesting to apply non-linear dimensionality
reduction techniques (e.g. principal curves, autoencoders) and evaluate the effects.

In our work, we padded an input sequence with zeros to enforce uniform input size. We proposed to pad our inputs by \textit{wrapping}
around the text to remove zero-padding. It is unclear from our experiments if this augmentation helped improve our model's performance directly.
We would like to extend our proposed padding to have a selection mechanism to pad with words that could potentially increase the information content in a training
sample (i.e. keywords). Having such a selection mechanism, we could pad an input sequence using words selected as informative or meaningful based on
some specified criteria rather than simply wrapping around the sequence.

Lastly, we are interested in seeing the effect of synonyms as a data augmentation procedure. This could be to replace words with their
actual synonyms using table lookups, or to compute nearest neighbors and replace words based on this criterion.
